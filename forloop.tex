\documentclass[10pt]{article}
\usepackage[margin=1in]{geometry}
\usepackage{pgffor}
\usepackage{pgfmath}
\usepackage{arrayjob}
\usepackage{hyperref}

\setlength\parindent{0pt}

\title{LaTeX for loop examples}
\author{Alice Toll}
\date{\today}

\begin{document}
\maketitle

\noindent Reference: \url{https://stuff.mit.edu/afs/athena/contrib/tex-contrib/beamer/pgf-1.01/doc/generic/pgf/version-for-tex4ht/en/pgfmanualse15.html}

\begin{verbatim}
\foreach \x in {1,2,3,0}{[\x]}
\end{verbatim}
\foreach \x in {1,2,3,0}{[\x]}
\bigskip

\begin{verbatim}
\foreach \x in {1,2,...,6} {\x, }
\end{verbatim}
\foreach \x in {1,2,...,6} {\x, }
\bigskip

\begin{verbatim}
\foreach \x in {1,2,3,...,6} {\x, }
\end{verbatim}
\foreach \x in {1,2,3,...,6} {\x, }
\bigskip

\begin{verbatim}
\foreach \x in {1,3,...,11} {\x, }
\end{verbatim}
\foreach \x in {1,3,...,11} {\x, }
\bigskip

\begin{verbatim}
\foreach \x in {1,3,...,10} {\x, }
\end{verbatim}
\foreach \x in {1,3,...,10} {\x, }

\begin{center}
\line(1,0){250}
\end{center}

Reference: \url{http://tex.stackexchange.com/questions/45040/pgf-tikz-how-to-store-strings-in-array} \\

\verb|\usepackage{pgfmath}|
\begin{verbatim}
\def\names{{"Jane", "Job", "Ben", "Alice"}}
\foreach \i in {0, ..., 3} {
	Name \i: \pgfmathparse{\names[\i]}\pgfmathresult \\
	}
\end{verbatim}
\def\names{{"Jane", "Job", "Ben", "Alice"}}
\foreach \i in {0, ..., 3} {
	Name \i: \pgfmathparse{\names[\i]}\pgfmathresult \\
	}

\verb|\usepackage{arrayjob}|
\begin{verbatim}
\newarray\names
\readarray{names}{Jane&Job&Ben&Alice}
\foreach \i in {1, 2, 3, 4} {
	\names(\i) \par
	}
\end{verbatim}
\newarray\names
\readarray{names}{Jane&Job&Ben&Alice}
\foreach \i in {1, 2, 3, 4} {
	\names(\i) \par
	}

\begin{center}
\line(1,0){250}
\end{center}

\foreach \name in {Jane, Job, Ben, Alice}
{
\section{About \name}
This section is about \name.
}

\begin{center}
\line(1,0){250}
\end{center}

%\newarray\names
%\readarray{names}{Jane&Job&Ben&Alice}
%\newarray\genders
%\readarray{genders}{female&male&male&female}
%\foreach \i in {1, 2, 3, 4} {
%	\section{About \protect{\names(\i)} }
%	This section is about \names(\i). \names(\i) is \genders(\i).
%	}



\def\names{{"Jane", "Job", "Ben", "Alice"}}
\def\genders{{"female", "male", "male", "female"}}
\foreach \i in {0, ..., 3} {
	\section{About \pgfmathparse{\names[\i]}\pgfmathresult }
	This section is about \pgfmathparse{\names[\i]}\pgfmathresult .
	\pgfmathparse{\names[\i]}\pgfmathresult is a \pgfmathparse{\genders[\i]}\pgfmathresult 
	}



\end{document}